                        %%%%%   PREAMBLE  %%%%%

\documentclass[11pt,spanish,a4paper,twoside]{article}

\usepackage[spanish,activeacute]{babel}
\usepackage{amssymb,amsfonts}
\usepackage{amsmath}             %  Para las f�rmula matem�ticas
\usepackage[latin1]{inputenc}    %  Permite escribir los acentos

\usepackage{fancyhdr}
\renewcommand{\headrulewidth}{0pt}  %  l�nea horiz. bajo el encabezado
\renewcommand{\footrulewidth}{0pt}  %  l�nea horiz. sobre el pie

\usepackage{makeidx}

% \usepackage{times}
% \usepackage[OT1,T1]{fontenc}

\usepackage{palatino}
% palatino: una fuente excelente, pero si sale
% mal en el .pdf descomentar las dos l�neas anterires
% y comentar �sta.

% \usepackage{charter}

\usepackage{vmargin}
\setpapersize{A4}
\setmarginsrb{2.5cm}{2cm}{2.5cm}{2cm}%
{1cm}{1cm}%
{1cm}{1cm}


\usepackage{graphicx}
\usepackage{color}       % Para el color de las fuentes
\usepackage{lettrine}    % Letra capital
\usepackage{enumerate}   % Entorno enumerate mejorado
\usepackage{mdwlist}     % Entorno descripcion mejorado
\usepackage[amsthm,hyperref,thref]{ntheorem}
%\usepackage{paralist}



\parskip 0.15cm          % Queda mejor pero mas paginas

                        %%%%%   HYPERREF  %%%%%

\usepackage{hyperref}        %  Hiperv�nculos
\hypersetup{                 %  Hiperv�nculos setup
    bookmarksopen=false,
    bookmarksnumbered=true,
    pdfpagemode=UseOutlines,  % None, UseThumbs, UseOutlines (show bookmarks), FullScreen
    colorlinks=true,     % color link text, not a box around them.
    linkcolor=blue,      % color for normal internal links.
    anchorcolor=black,   % color for anchor text.
    citecolor=blue,      % color for bibligraphical citations in text.
    filecolor=black,     % color for URLs which open local files.
    menucolor=black,     % color for Acrobat menu items.
    pagecolor=black,     % color for links to other pages.
    urlcolor=blue,      % color for linked URLs.
    breaklinks=false,    % Allows link text to break across.
    pdfstartview={FitBH},
    pdfview={FitBH},
    %hyperindex=true,
    pageanchor=false
}

% Para el �ltimo
% En realidad, no pasa eso, pero si no pongo "false" tira errores.
%    Determines whether every page is given an implicit
%    anchor at the top left corner. If this is turned off,
%    \tableofcontents will not contain hyperlinks.

% El pen�ltimo me tira un warning pero dejarlo sin comentar     % Hiperreferencias

\usepackage{url}

\usepackage{longtable}
\usepackage{supertabular}


\usepackage{tikz}
\usepackage{pgf}
\usepackage{wrapfig}


\usepackage{algorithm2e}
\usepackage{algpseudocode}