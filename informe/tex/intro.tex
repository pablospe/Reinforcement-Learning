\section{Introducci�n}
\label{Intro}

\noindent
Un agente (una persona, un animal, un robot o un programa) necesita tomar decisiones en un ambiente desconocido. Sus decisiones tienen consecuencias inmediatas, de mediano y de largo plazo. En todos los casos, se desea maximizar alg�n tipo de beneficio (comida, trabajo realizado, paquetes enviados) y minimizar  alg�n costo (tiempo, dinero, energ�a). Ante lo incierto del entorno y del resultado esperado de las acciones a tomar, el agente tiene que decidir si actuar en base al conocimiento que tiene, o invertir tiempo en explorar y adquirir m�s informaci�n.

En los '80 y '90, ideas que ven�an de la psicolog�a, teor�a de control, optimizaci�n y microeconom�a dieron nacimiento al campo de RL. Unos de sus fundadores, Rich Sutton (junto con Andy Barto), dec�a: \textit{�la hip�tesis del aprendizaje por refuerzos es que aquello que entendemos por comportamiento inteligente surge de un agente que intenta maximizar la esperanza a largo plazo de la suma de \textit{\textbf{refuerzos}} recibidos (usualmente en un ambiente desconocido)�}. La idea es que nuestros agentes aprendan a comportarse de manera (cuasi-)�ptima solamente guiados por su af�n de maximizar una se�al de refuerzo, pero sin la presencia de un experto que les indique qu� acciones tomar en cada momento.

El formalismo de RL ha tenido un �xito significativo en t�rminos de algoritmos y aplicaciones que produjo en sus 25 a�os de vida. En rob�tica, donde tuvo un especial �xito, aplicaciones de RL incluyen helic�pteros a control remoto que pueden volar solos y hasta en forma invertida, robots cuadr�pedos que sortean obst�culos complejos, perros robots que encuentra la manera m�s r�pida posible de correr. RL tambi�n tuve un fuerte impacto en juegos como el backgammon, donde el algoritmo TD-Gammon logr� derrotar a los mejores jugadores humanos, y hoy en d�a est� teniendo mucho �xito en juegos como el Go, con programas como RLGo que juegan en tableros de $9$\texttt{x}$9$ a nivel de Masters humanos. Los casos anteriores, sin embargo, cuentan parte de la historia. RL recibe muchas cr�ticas: que no es escalable, que no es confiable, que los algoritmmos tardan demasiado en converger.

Este trabajo est� organizado de la siguiente manera: secci�n 2, trata el formalismo est�ndar de RL; secci�n 3, provee una explicaci�n b�sica y con ejemplos del Q-Learning algoritmo; secci�n 4, se ocupa de RL-glue y RL-library (proyectos open-source que permitieron la implementaci�n del algoritmo); secci�n 5, habla de la implementaci�n en s�, usada para el tetris; y por �ltimo, las conclusiones.